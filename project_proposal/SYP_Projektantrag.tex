\documentclass[]{article}

%opening
\title{Hauer On Repeat}
\author{Ida Hönigmann \and Manuel Kubu}

\begin{document}

\maketitle
\newpage
\tableofcontents
\newpage

\section{Vorarbeit}
Im Schuljahr 2018/19 wurde bereits im Rahmen des Projekts \mbox{"Hauer On Repeat"} ein Exponat im Stadtmuseum Wiener Neustadt im Rahmen der Landesausstellung 2019 aufgestellt. Der Projektauftraggeber war die Josef Matthias Hauer Musikschule Wiener Neustadt.

Das Exponat ermöglicht den Benutzern durch Eingabe von zwölf Zahlen ein eigenes Zwölftonspiel zu erstellen und sich dieses mitzunehmen.

\subsection{Bestehendes Exponat}
Das Exponat besteht aus einem Raspberry Pi, auf dem die von uns geschriebene Software läuft, einer Konstruktion aus MDF, einem Monitor, zwei Einhand-Kopfhörer und 12 Taster mit eingebauten LEDs.

Die Software besteht aus mehreren Teilen:
\begin{itemize}
	\item Implementierung des Algorithmus von Josef Matthias Hauer (C++)
	\item Graphical User Inferface (HTML, CSS und Javascript)
	\item Speichern in Datenbank (Java)
	\item Schnittstelle zu Twitter (Java)
	\item Startup Script (Shell)
\end{itemize}

\subsection{Weiterverwendung des bestehenden Exponats}
Nach Ende der Landesausstellung 2019 muss überlegt werden welche Teile des Exponats abgekauft / übernommen werden können.

Annahme was weiter verwendet wird.

\section{Ziele}

\section{Projektteam}

\section{Projektumfang}


\end{document}
