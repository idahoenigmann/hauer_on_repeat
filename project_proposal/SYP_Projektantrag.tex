%Ida Hoenigmann, Mathias Eitler, Manuel Kubu

%4robert@gmx.at

\documentclass[a4paper]{scrartcl}

\usepackage{graphicx}
\usepackage[utf8]{inputenc}
\usepackage{geometry}
\usepackage{gensymb}
\usepackage{overpic}
\usepackage{anyfontsize}
%\usepackage[german]{babel} %Ich bekomme hier einen Fehler

\setcounter{tocdepth}{1}

%opening


\begin{document}

\begin{titlepage}
	\newgeometry{margin=0in}
	\noindent
	\begin{overpic}[height=\textheight,width=\textwidth]{Pictures/logo_hauer_on_repeat_w_background}
		\put (13,85) {\fontsize{40}{48}\selectfont HAUER ON REPEAT 2.0\fontsize{15}{18}\selectfont}
		\put (13,80) {\fontsize{15}{18}\selectfont Erweiterung zum interaktiven Exponat zur Generierung eines Zwölftonspiels}
		\put (13,78) {\fontsize{15}{18}\selectfont nach Josef Matthias Hauer}
		\put (13,20) {\fontsize{20}{24}\selectfont Autor: Mathias Eitler}
		\put (13,14) {\fontsize{20}{24}\selectfont 14.05.2019}
		\put (13,10) {\fontsize{15}{18}\selectfont Auftraggeber: Robert Michael Weiß/J. M. Hauer Musikschule}
	\end{overpic}
\end{titlepage}

\newpage
\tableofcontents
\newpage

\section{Projektinformationen}

\textbf{Projektbezeichnung: }Hauer on repeat 2.0

\noindent \textbf{Projektkurzbeschreibung: }Ziel dieses Projekts ist es das interaktive Exponat zu Hauers Zwölftonspiel zu erweitern, welches noch auf der Landesausstellung 2019 und danach für kurze Zeit in der Muk in Wien sowie auf weiteren Standorten präsentiert wird.

\noindent \textbf{Antragsteller: }Ida Hönigmann

\noindent \textbf{Auftraggeber: }Robert Michael Weiß, Josef Matthias Hauer Musikschule

\section{Projekthintergrund}
Im Schuljahr 2018/19 wurde bereits im Rahmen des Projekts \textbf{Hauer On Repeat} ein Exponat im Stadtmuseum Wiener Neustadt im Rahmen der Landesausstellung 2019 aufgestellt. Der Projektauftraggeber ist die Josef Matthias Hauer Musikschule Wiener Neustadt.
\newline Das Exponat ermöglicht den Benutzern durch Eingabe von zwölf Zahlen ein eigenes Zwölftonspiel zu erstellen und sich dieses mitzunehmen.

\section{Projektidee}
Der Besucher soll bei der Erzeugung eines Zwölftonspieles neue Features erleben. Einerseits das Hochladen der eigens kreierten Musik, anderseits das Anzeigen der Noten während die Taster betätigt werden. Die Zwölftonfolge soll zusätzlich durch Hauers eigenen Spielregeln erweitert werden und somit in klingende und visualisierte Musik umgesetzt.

\subsection{Bestehendes Exponat}
Das Exponat besteht aus einem Raspberry Pi, auf dem die von uns geschriebene Software läuft, einer Konstruktion aus MDF, einem Monitor, zwei Einhand-Kopfhörer und zwölf Taster mit eingebauten LEDs.\newline
Die Software besteht aus mehreren Teilen:
\begin{itemize}
	\item Implementierung des Algorithmus von Josef Matthias Hauer (C++)
	\item Graphical User Inferface (HTML, CSS und Javascript)
	\item Startup Script (Shell)
\end{itemize}

\subsection{Weiterverwendung des bestehenden Exponats}
Nach Ende der Landesausstellung 2019 muss überlegt werden welche Teile des Exponats abgekauft / übernommen werden können. Diesbezüglich verhandelt Herr Weiß noch mit der Landesausstellung.
\newline Annahme was weiter verwendet wird:
\begin{itemize}
	\item Die zwölf Taster
	\item Der Raspberry PI
	\item Die Lakrylplatte
\end{itemize}

\section{Projektziele}
\subsection{Funktionale Ziele}
Das Endergebnis des Projekts besteht aus den folgenden Teilzielen:
\begin{itemize}
	\item \textbf{Twitter-API} \newline
	Das Fertigstellen der Twitter-API, welche dazu dient, dass am Ende eines Durchganges ein QRCode erscheint, welcher den User zu seinem erstellten Song führt, ist eines der wichtigen Ziele. Dabei wird der Song auf dem offiziellen "Hauer on Repeat"-Account hochgeladen und dort abspielbar sein. Die Schnittstelle wird hierbei in Java realisiert.

    \item \textbf{Speichern in die Datenbank} \newline 
	Das Speichern der verschiedenen Button-Reihenfolgen in die Datenbank ist ebenfalls ein essentieller Punkt in der Erweiterung des Projektes. Dabei wird jede Reihenfolge, die noch nicht eingegeben wurde, frisch in die Datenbank abgelegt und mit einem zusätzlichen Attribut die Anzahl der eingegebenen mit gespeichert. Hierbei wird das Abspeichern ebenfalls in Java realisiert. 
	
	\item \textbf{Sofortiges Anzeigen der Noten} \newline
	Soweit der Besucher die Taster betätigt, werden schon die dazugehörigen und passenden Noten auf dem Monitor angezeigt, bis dieser alle gedrückt hat.  
	
	\item \textbf{Globale Vernetzung} \newline 
	Es soll die Möglichkeit bestehen, von mehreren Exponaten die Daten bezüglich des Drückens der selben Reihenfolge global abspeichern und verwalten zu können. 
	
	\item \textbf{Mobilität} \newline 
	Es soll eine mobile Variante vom Exponat gebaut werden, die dann für die Transporte zu den einzelnen Standorte verwendet wird. Somit soll ein leichterer Aufbau und Abbau ermöglicht werden, die bei den verschiedenen Ausstellungen getätigt werden müssen. Diese mobile Version soll auch mehrmals produziert werden, damit das auch im Ausland einsatzbereit wäre. Natürlich werden wir sie nicht bauen, sondern ein externes Unternehmen wird diese dann für uns anfertigen.
	
\end{itemize}

\subsection{Qualitätskriterien / nicht-funktionale Ziele}
Bei der Durchführung des Projekts muss auf folgende Kriterien geachtet werden:
\begin{itemize}
	\item Exponat muss widerstandsfähig und damit dem rauen Ausstellungsbetrieb gewachsen sein
	\item Exponat muss leicht und schnell wartbar sein
	\item Exponat muss für alle Altersgruppen geeignet sein
	\item Bedienung muss einfach zu verstehen sein
\end{itemize}

\section{Termine}

\subsection{Projektstart}
Das Projekt wird im September 2019, was dem Beginn des Schuljahres 2019/20 entspricht, beginnen.

\subsection{Meilensteine}
\begin{center}
	\begin{tabular}{|l|l|l|l|}
		\hline
		\textbf{Nummer} & \textbf{Meilenstein} & \textbf{Soll} & \textbf{Ist} \\
		\hline
		0 & Externe Koordination (Expertise-Treffen mit R. M. Weiß) & 2.10.2019 & -\\
		1 & Einbau von Twitter-API & 31.10.2019 & -\\
		2 & Einbau der Datenbank & 31.10.2019 & -\\
		3 & Erweiterung des Zwölftonspiels & 6.11.2019 & -\\
		4 & Design für mobile Version überlegen & 30.11.2019 & -\\
		5 & Mobile Version bauen lassen & 10.2.2020 & -\\
		6 & Allgemeine Design-Verbesserungen & - & -\\
		\hline
	\end{tabular}
\end{center}


\section{Projektorganisation}
\subsection{Kunden/Beteiligte}
%Das Projekt wird für die Josef Matthias Hauer Musikschule durchgeführt. 
Das Projekt wird für Robert M. Weiß/Josef Matthias Hauer Musikschule für die Landesausstellung 2019 sowie für die Muk und weiteren Ausstellungen durchgeführt. 

\noindent Die Durchführung wird von Ida Hönigmann, Mathias Eitler und Manuel Kubu vorgenommen.

\noindent Benutzer des Exponats sind die Besucher der NÖ Landesaustellung 2019, Muk und Grünbach.

\subsection{Projektteam}
Projektleiterin: Ida Hönigmann

\begin{center}
	\begin{tabular}{|l|l|l|l|l|}
		\hline
		\textbf{Person} & \textbf{Projektmanagement} & \textbf{Design} & \textbf{Programmieren} \\
		\hline
		Ida Hönigmann & 2,5 & 12,5 & 20 \\
		Mathias Eitler & 2,5 & 12,5 & 20 \\
		Manuel Kubu & 2,5 & 12,5 & 20 \\
		\hline
	\end{tabular}
\end{center} 
Das gesamte Projektteam besucht die HTL Wiener Neustadt.

\section{Stakeholderanalyse}
\begin{center}
	\begin{tabular}{|l|l|l|l|l|}
		\hline
		\textbf{Umfeldgruppe} & \textbf{Machteinfluss} & \textbf{Befürchtungen} & \textbf{Erwartungen} & \textbf{Vorkehrungen} \\
		\hline
		Projektteam & viel Einfluss & Arbeitsüberforderung &  - & \\
		\hline
		Musikschule & viel Einfluss & - & hohe Erwartung auf & - \\
		& & & das Projektergebnis &\\
		\hline
		Landes- & viel Einfluss & Bürokratie & - & frühe Kontakt-\\
		ausstellungs- & & & &aufnahme mit den\\
		verantwortlicher& & & & Verantwortlichen\\
		\hline
		Besucher & wenig Einfluss & Vandalismus, & Unterhaltung & leicht zu bedienen\\
		& & Inkompetenz mit IT & &\\
		\hline
		Herr Weiß & viel Einfluss & - & Unterstützung & - \\
		\hline
		Muk & viel Einfluss & - & - & - \\
		\hline
		Grünbach & viel Einfluss & - & - & - \\
		\hline
	\end{tabular}
\end{center}

\section{Projektumfang}
Der Projektumfang wird sich hierbei um Erweiterungen und / oder Veränderungen am Exponat beziehen. 
\newpage
\section{Projektrisiken}
Bei nicht erfolgreichem Projektabschluss muss entweder ohne jegliche Erweiterung das Exponat in der Muk und in Grünbach aufgestellt werden, was unseren Projektauftraggeber enttäuschen würde. Obendrauf würde dies einen schlechteren Ruf auf die HTL Wiener Neustadt werfen, da dann exakt dasselbe Exponat wieder neu aufgestellt werden würde. 

\vspace{2cm}................................................................\hspace{2cm}................................................................

Robert M. Weiß\hspace {6cm}Ida Hönigmann

\vspace{1cm}................................................................
	
\end{document}